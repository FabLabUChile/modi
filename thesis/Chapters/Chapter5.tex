% Chapter Template

\chapter{Aplicaciones} % Main chapter title

\label{ChapterX} % Change X to a consecutive number; for referencing this chapter elsewhere, use \ref{ChapterX}

\lhead{Capitulo 4. \emph{Aplicaciones}} % Change X to a consecutive number; this is for the header on each page - perhaps a shortened title

%----------------------------------------------------------------------------------------
%	SECTION 1
%----------------------------------------------------------------------------------------

\section{Educación}

Un enjambre de robots puede presentar muchas ventajas dentro del aula. Si se tiene un sistema de fácil uso para los alumnos, el profesor puede asignar una tarea a un grupo de estudiantes donde cada uno tiene la responsabilidad de controlar o programar un robot para que el conjunto logre una meta determinada como ordenar unos bloques o hacerse cargo de regar un pequeño huerto. Abusando un poco del concepto de la colectividad, incluso pueden generarse tareas donde cada colegio se especializa en un tipo de tareas para luego juntar los distintos robots y probar cómo interactúan.

Tener un setup con robots que demuestren un comportamiento colectivo puede ser muy ventajoso para promover el aprendizaje de roles sociales si la docente a cargo programa los robot para que tengan distintos roles como Lider y participante. Este mismo juego de roles puede ayudar a los niños a generar estrategias y herramientas para enfrentarse al mundo. En síntesis haciendo juegos para los niños con los robots se puede tener su atención para reforzar su educación y favorecer en el desarrollo de habilidades y competencias. 

\section{Usos Militar}

Con un enjambre de robots, se puede simular una situación de catástrofe, donde es necesario poder desplegar un grupo de robots para hacer una tarea de reconocimiento y así poder buscar personas heridas o atrapadas.
Imaginemos una situación hipotética donde un edificio es destruido, la búsqueda de sobrevivientes no es una tarea fácil, implica que rescatistas ingresen al lugar corriendo grave peligro, usualmente buscando a las víctimas en condiciones de poca visibilidad. Esto mismo podría ser ejecutado por un enjambre robótico que esté programado para buscar gente y que de manera colectiva recorra un área mucho mayor que 2 o 3 personas. Incluso un robot del mismo enjambre puede fallar, pero al ser un sistema distribuido el enjambre continúa funcionando, es un sistema muy robusto.

\section{Usos Doméstico}

Existen las aspiradoras Roomba, que sin necesidad de un operario humano pueden aspirar nuestras casas. Ellas recorren nuestro hogar y gracias a sus sensores pueden auto generar un mapa del entorno. Este modelo del hogar puede hacerse mucho más rápido si en vez de tener un solo robot, se tienen 20. El mismo concepto se puede aplicar en seguridad del hogar, donde se pueden tener varios mini robots haciendo rondas en el perímetro de nuestra casa y en forma colectiva abarcan lo más posible. Todas estas situaciones se pueden simular con los robot MODI.
