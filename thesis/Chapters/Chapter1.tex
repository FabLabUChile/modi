% Chapter 1

\chapter{Introducción} % Introduccion al trabajo

\label{Chapter1} % For referencing the chapter elsewhere, use \ref{Chapter1} 

\lhead{Capítulo 1. \emph{Introducción}} % This is for the header on each page - perhaps a shortened title

%----------------------------------------------------------------------------------------
%--------------------------------------------------------
%Sección 1
%--------------------------------------------------------


\begin{figure}[htbp]
	\centering
		\includegraphics[width=0.8\textwidth]{./Figures/robot.jpg}
		%\rule{35em}{0.5pt}
%	\caption[Robot Huggable]{Robot creado por MIT Media Lab para cuidados personales. Imagen tomada de \cite{Stiehl:2006:HTR:1179133.1179149}}
	\label{fig:Huggable}
\end{figure}

La robótica es una rama de la tecnología que cada vez toma más importancia. Gracias a sus avances podemos construir complejas máquinas que nos asisten en tareas imposibles para nosotros los seres humanos y también nos permite estudiar la naturaleza al poder reproducir ciertos aspectos de ella con estas maquinas. Un aspecto que ultimamente a llamado la atención de los investigadores es, como los sistemas complejos existente en la naturaleza son capacez de llevar a cabo tareas que cada uno de sus individuos de forma aislada no pueden realizar. Actualmente existen investigadores que por medio de robots simples han logrado reproducir comportamientos colectivos como los observados en hormigas y otros animales. En Chile aún hay muy pocos grupos de investigación que centren sus trabajos en este tema, y es necesario contar con robots de bajo costo que faciliten la construcción de grupos de robots.

El trabajo se centró en la investigación de las tecnologías existentes para diseñar un robot que sea económico, simple y que de manera inalámbrica permita su control. El robot debe tener estas características ya que se necesita construir un enjambre para poder realizar estudios en el comportamiento y control de éstos.

Al comienzo de este proyecto se trabajó en Santiago de Chile, en conjunto con la Universidad de Chile (UChile) bajo la tutela del Doctor Juan Cristóbal Zagal en el Laboratorio de Síntesis de Máquinas Inteligentes. Por parte de la Universidad Técnica Federico Santa María (USM), se trabajó con la Doctora María José Escobar, del Departamento de Electrónica y luego se incorporó el Doctor Pablo Prieto del Departamento de Diseño de Productos. El financiamiento para realizar este prototipo ha sido por parte de las dos Instituciones, Universidad Técnica Federico Santa María y Universidad de Chile.

Durante el proceso de desarrollo se tuvieron que hacer diversas compras de materiales, pero los lugares más recurrentes al momento de hacerlas fueron Olimex y Casa Royal. La primera es una empresa dedicada a traer productos para hacer prototipos y construir máquinas, la segunda cuenta con varios insumos básicos para trabajar en desarrollo de circuitos electrónicos. Ambas empresas se encuentran en Santiago, por lo que trabajar en esta ciudad es de gran ayuda para reducir los tiempos en desarrollo.
Luego de armar un primer robot funcional, con materiales disponibles en Santiago de Chile, se hizo una búsqueda en internet de componentes en tiendas especializadas.

Este documento resume el desarrollo del proyecto MODI (Modular Intelligence), que es un robot simple de construir que tiene como fin ser repetible para facilitar la construcción de Enjambres de robots. En el capítulo 2 se explica lo que es un robot, algo de historia y algunos tipos de robots. El capítulo 3 comienza con un explicación sobre los enjambres de animales, luego se describen los robots usados actualmente en la literatura para estudiar enjambres y al final se describen las necesidades de mercado y las ventajas del proyecto MODI. El capítulo 4 está centrado en el diseño y herramientas de fabricación utilizadas durante el proceso de desarrollo. El capítulo 5 es el principal, describe criterios de diseño y el detalle de todos los componentes utilizados en el robot diseñado. Además se explica el software utilizado y el sistema que permite hacer un seguimiento de cada robot de forma individual. En el capítulo 6 se explican algunos posibles usos de los enjambres de robots. Al final, el capítulo 7 tiene sugerencias para mejoras futuras y las conclusiones. 

Para el comienzo de cada capítulo se escogío una imagen representativa que, por la diagramación escogida van sin leyenda. Estas imagenes son: Capítulo 1. Huggable, robot desarrollado por MIT para el cuidado y educación de niños; Capítulo 2. Digesting Duck, imagen perteneciente al dominio publico; Capítulo 3. Bandanda de Auklet, imagen perteneciente al dominio publico; Capítulo 4. Función matemática madera, diseñada por FabLab Barcelona; Capítulo 5. Robot MODI, robot diseñado en este trabajo.


%----------------------------------------------------------------------------------------