% Chapter Template

\chapter{MODI} % Main chapter title

\label{Chapter3} % Change X to a consecutive number; for referencing this chapter elsewhere, use \ref{ChapterX}

\lhead{Capítulo 3. \emph{MODI}} % Change X to a consecutive number; this is for the header on each page - perhaps a shortened title

%----------------------------------------------------------------------------------------
%	SECTION 1
%----------------------------------------------------------------------------------------

\section{Diseño}

\begin{figure}[htbp]
	\centering
		\includegraphics[width=0.8\textwidth]{./Figures/MODI/render.png}
		\rule{35em}{0.5pt}
	\caption[Robot MODI]{robot MODI (sigla para Modular Intelligence)}
	\label{fig:MODI}
\end{figure}

La función principal de MODI es ser una plataforma móvil de fácil acceso. Existe un repositorio en GitHub  \footnote{https://github.com/FabLabUChile/modi} donde se tienen los códigos actualizados para controlar y construir robots MODI. Una lista con componentes extras para comprar se encuentra en la Figura \ref{fig:BOM}.
\linebreak 

\textbf{Requerimientos de desarrollo}

\begin{itemize}
\item Simple de fabricar
\item Open Source
\item Fabricable en un Fab Lab
\item Componentes extras necesarios, de fácil acceso
\item Minimizar la cantidad de componentes mecánicos extras necesarios
\item Carga Autónoma con celda solar
\item Seguimiento de grupo
\item Control individual del color de cada MODI
\item Movimiento simple de cada robot de forma independiente
\end{itemize}

%-----------------------------------
%	SECTION 2
%-----------------------------------

\section{Construcción}

Construir un robot implica el diseño en computador de varios componentes. Estos pueden ser circuitos electrónicos (PCB), software de control y piezas mecánicas. Uno de los requerimientos principales del desarrollo es que sea simple de fabricar, es por esto que en el resultado final se reflejan las horas de diseño.

%-----------------------------------
%	SUBSECTION 2
%-----------------------------------
\subsection{Fabricación Digital vs Análoga}
Cuando se quiere pasar una idea al mundo real es necesario un proceso de fabricación. Dependiendo de la cantidad de herramientas que se tenga es más o menos fácil la tarea. Desde el comienzo hasta hace un par de años, quienes se dedican a construir robots, debían construir de manera \textit{artesanal} donde es imposible que las piezas queden todas iguales y el tiempo empleado era bastante. Hoy en día existe una gran alternativa que surge como un nuevo paradigma, la Fabricación Digital. Las impresoras 3D, que no son más que un extrusor montado en un sistema con ejes que le dan 3 grados de libertad, permiten desde un modelo en un computador, obtener un objeto real hecho en plástico. También existe otro tipo de máquinas que permite hacer diseños en 2D, estas son las cortadoras LASER. La primera versión de MODI fue construido usando planchas de madera MDF y acrílico, cortados en LASER.

La primera versión se contruyó utilizando técnicas tradicionales, Figura \ref{fig:analgTodigital} a y utilizando una impresora 3D Makerbot Replicator 1 se hizo la primera versión con piezas plásticas,Figura \ref{fig:analgTodigital} b .

\begin{figure}[htbp]
	\centering
		\includegraphics[width=0.8\textwidth]{./Pictures/modi_analogToDigital.png}
		\rule{35em}{0.5pt}
	\caption[Comparación de construcción análoga y digital]{\textbf{a.} Primera versión MODI construido con MDF y bastante silicona, \textbf{b.} MODI usando técnica de \emph{ Fabricación Digital }con chasis de plástico construido con una MakerBot Replicator 1}
	\label{fig:analgTodigital}
\end{figure}

Esta versión presentaba el problema de involucrar demasiadas partes que debían ser hechas por una persona. El microcontrolador, junto con la radio inalámbrica se colocaron en una placa electrónica para prototipado, más adelante se puede ver que fueron reemplazados por una PCB llamada Arduino FIO, que es una plataforma de desarrollo Arduino junto con un socket Xbee. Otro factor clave para descartar esta versión es que utiliza 4 pilas AAA que necesitan ser removidas para poder ser recargadas, lo que impide que en futuras versión exista la posibilidad de una carga autónoma por parte de los Robots. La versión actual de MODI permite su carga por medio de un puerto Mini USB, panel solar y de forma inalámbrica.

Aunque han bajado los precios de las maquinas para prototipado rápido, aún no están al alcance de todas las personas. Es por esto que existen los Fab Labs (acrónimo del inglés Fabrication Laboratory), que según Wikipedia es, \textit{"un espacio de producción de objetos físicos a escala personal o local que agrupa máquinas controladas por ordenadores. Su particularidad reside en su tamaño y en su fuerte vinculación con la sociedad."} Los Fab Labs están por todo el mundo, Figura \ref{fig:Fablabs}. MODI fue concebido como un proyecto del Fab Lab de la Universidad de Chile y por esto es posible reproducirlo en cualquier Fab Lab. En Chile, además del Fab Lab de la Universidad de Chile están: Design LAB UAI, Fab Lab Santiago y Stgo MakerSpace. 

\begin{figure}[htbp]
	\centering
		\includegraphics[width=\textwidth]{./Figures/map.png}
		\rule{35em}{0.5pt}
	\caption[Mapa Fab Labs en el mundo.]{Mapa actual de lugares en el mundo que cuentan con un Fab Lab. En Chile a la fecha existen 3. Imagen tomada de fablabamersfoort.nl/fablabs/}
	\label{fig:Fablabs}
\end{figure}	

%-----------------------------------
%	SUBSECTION 3
%-----------------------------------
\subsection{Software CAD}

CAD viene de sus siglas del Inglés, Computer-aided design, y se refiere a un diseño asistido por herramientas computacionales. Profesionales como ingenieros, arquitectos y del área del diseño por lo general son los que hacen más uso de estas herramientas.

Parte de la definición de Wikipedia es, \textit{”... se pueden dividir básicamente en programas de dibujo en dos dimensiones (2D) y modeladores en tres dimensiones (3D). Las herramientas de dibujo en 2D se basan en entidades geométricas vectoriales como puntos,líneas, arcos y polígonos, con las que se puede operar a través de una interfaz gráfica....”}

Durante el transcurso del proyecto se trabajó con varios softwares CAD. El primero fué SketchUp 8 de Google, que permite fácilmente hacer bocetos de lugares y cuenta con una importante biblioteca de modelos para incluir en el diseño. Rápidamente se pudo hacer un scketch utilizando modelos descargados de Internet , Figura~\ref{fig:setup}.

El diseño del chasis junto con las ruedas y demás partes plásticas, se hizo en un comienzo con SolidWorks 2012 y luego por ser más simple de usar, Inventor 2013 de Autodesk. Ambos softwares permiten generar modelos en 3D para luego exportar el diseño al formato STL que es estándar para prototipar en plástico. Luego de varias iteraciones se lográ el modelo final, Figura \ref{fig:compRender} b, que tiene cinco piezas. 


\begin{figure}[htbp]
	\centering
		\includegraphics[width=\textwidth]{./Figures/MODI/compRender.png}
		\rule{35em}{0.5pt}
	\caption[Comparación entre el primer render realizado y el último]{\textbf{ a.} Primer Chasis de MODI, realizado con SolidWorks 2012. Esta versión sirve solo como prueba de concepto ya que no tiene espacio para las conexiones eléctricas necesarias.\textbf{ b.} Última versión de MODI a la fecha, el diseño se realizó en Inventor 2013 y el render se obtuvo con KeyShot 4.}
	\label{fig:compRender}
\end{figure}	





%-----------------------------------
%	SUBSECTION 4
%-----------------------------------
\subsection{Impresora 3D} \label{cap:impresora3D}

Parte importante de este trabajo se realizó con impresoras 3D. Estas existen desde los años 80' y hasta hace algunos años por su precio y tamaño eran de difícil acceso. Utilizan distintas tecnologías, las que se utilizaron en el proceso de este trabajo son las FDM, sigla del ingles "Fused deposition modeling" donde un cabezal extrusor de plástico se mueve en el plano xy depositando capa por capa de material mientras avanza en el eje z.

\begin{figure}[htbp]
	\centering
		\includegraphics[width=\textwidth]{./Figures/3Dprint.png}
		\rule{35em}{0.5pt}
	\caption[Modelo funcionanmiento Impresora 3D]{Modelo de funcionamiento impresoras 3D MDF. Tiene dos materiales, uno para construir la pieza y otro que funciona como soporte estructural para la impresión, además existe una base que se calienta para sujetar la pieza mientras se construye.}
	\label{fig:3Dprint}
\end{figure}	

Primero se utilizó la 3D Touch Dual de la empresa Bit from Bytes, que permite hacer modelos de hasta 185 x 273 x 200[mm], con una buena resolución de 0.125[mm]\footnote{teambastech.com/Store/index.php?route=product/product\&product\_id=205} cada capa en eje z. Luego por facilidad de uso y menor tiempo de impresión se usa una impresora MakerBot Replicator 1

\begin{figure}[htbp]
	\centering
		\includegraphics[width=\textwidth]{./Figures/tabla_impresoras.png}
		\rule{35em}{0.5pt}
	\caption[Tabla comparativa de impresoras 3D]{Comparación de las caracteristicas mas importantes de impresoras 3D, Replicator 1, 3D Touch y Uprint}
	\label{fig:TablaImpresoras}
\end{figure}


\begin{figure}[htbp]
	\centering
		\includegraphics[width=\textwidth]{./Figures/impresoras.png}
		\rule{35em}{0.5pt}
	\caption[Impresoras 3D]{\textbf{a.} 3D Touch Dual de la empresa Bit from Bytes, \textbf{b.} Uprint SE de Stratasys, \textbf{c.} Replicator 1 de Makerbot, impresora 3D FDM utilizada para prototipar chasis y ruedas de robot MODI, \textbf{d.} Replicator 2 de MAkerbot, tiene mejor resolución que la versión 1.}
	\label{fig:impresoras}
\end{figure}	
\FloatBarrier

%-----------------------------------
%	SUBSECTION 5
%-----------------------------------
\subsection{Diseño PCB}

En los primeros prototipos un problema que se repetía constantemente son las conexiones entre los sistemas que componen al robot. Son 15 conexiones que unen distintos componentes, Figura \ref{fig:Diagrama cables}. Por esto se requiere un PCB que permita reducir al mínimo el espacio necesario para energizar y comunicar los componentes. El requerimiento principal es poder fabricarlo con una fresa CNC como la Roland Modela MDX-20, por lo que para facilitar el proceso de construcción es necesario diseñar un PCB de una sola capa. Se utilizó Eagle como software para el diseño del diagrama eléctrico y PCB, el resultado se puede ver en la Figura \ref{fig:pcb}.


\begin{figure}[htbp]
	\centering
		\includegraphics[width=\textwidth]{./Figures/Diagrama.jpg}
		\rule{35em}{0.5pt}
	\caption[Diagrama de conexiones eléctricas necesarias para unir todos los componentes]{Son 15 conexiones que unen distintos los componente.}
	\label{fig:Diagrama cables}
\end{figure}	

\begin{figure}[htbp]
	\centering
		\includegraphics[width=0.9\textwidth]{./Figures/modi/pcb.png}
		\rule{35em}{0.5pt}
	\caption[Fabricación de PCB]{\textbf{a.} Roland Modelo MDX-20 usada para hacer PCB y modelos 3D \textbf{b.} MODIBoard es una PCB diseñada para poder conectar de forma simple los diversos componentes electrónicos. Conecta: Arduino Fio, Lipo Rider Pro, Motor Driver, LED RGB y Motores, \textbf{c. }PCB fresada en dos maquinas diferentes. Se puede apreciar en el detalle aumentado 60x, la importancia de contar con una buena resolución de fresado. En especial es necesario para tener un buen pad para facilitar el soldado.}
	\label{fig:pcb}
\end{figure}

%-----------------------------------
%	SECTION 2
%-----------------------------------

\section{Componentes electrónicos}

\begin{figure}[htbp]
	\centering
		\includegraphics[width=\textwidth]{./Figures/MODI/compElo.png}
		\rule{35em}{0.5pt}
	\caption[Componentes Electrónicos]{Todos los componentes electrónicos que forman MODI \textbf{ a.} Batería LIPO 2,000 mAh, \textbf{b.} XBee, \textbf{c.} LED Rgb difuso, de esta manera se obtiene un ángulo de visión mucho más amplio, \textbf{d.} Arduino Fio, \textbf{e.} Lipo Rider Pro,\textbf{ f.} Motor DC 30:1 \textbf{g.} Motor Driver 1A Dual TB6612FNG.}
	\label{fig:compELO}
\end{figure}

MODI está compuesto por varios componentes electrónicos que implementan tecnologías nuevas. De forma sencilla el usuario puede hacer uso de todas estas, que por ser Open Source tienen una amplia documentación. Su uso en colegios, talleres y universidades, por ser de bajo costo, y sin partes externas extremadamente frágiles, es una excelente herramienta para iniciar a alumnos en temas como la Programación, Robótica, Comunicación Inalámbrica y Visión Artificial, entre otros, sin los miedos asociados a utilizar un equipo costoso y delicado.

El robot MODI es pequeño y de fácil construcción con las herramientas de Fabricación Digital que se puede encontrar en un FabLab. Tiene un pequeño microcontrolador \textbf{Arduino} que controla dos motores, y un \textbf{XBee} para la comunicación inalámbrica con un computador central que usando cualquier lenguaje de programación con comunicación serial puede controlar uno o más robots. La posición de cada robot se obtiene con tags \textbf{Fiduciales}, que permiten por medio de una cámara cenital saber la coordenada y rotación de cada uno. En esta primera aplicación es necesario contar con autonomía energética por lo que se incluye una batería lipo de 2000[mAh] junto a un módulo que carga la batería y eleva el voltaje de salida a 5[v]. Este el módulo de carga, \textbf{Lipo Rider Pro}, tiene además la ventaja de permitir distintos tipos de carga, como una celda de carga solar o un sistema de carga inalámbrica por medio de bobinas. Tener la posibilidad de cargar una batería con luz solar o por medio de forma inalámbrica hace posible el hacer experimentos que necesiten estar funcionando semanas sin la intervención de un humano para cargar cada uno.

%-----------------------------------
%	SUBSECTION 1
%-----------------------------------
\subsection{Arduino FIO}
El Arduino FIO, Figura \ref{fig:compELO} d, ha sido diseñado por Shigeru Kobayashi, es una placa para microcontrolador basada en el ATmega328P. Funciona a 3.3V y 8 MHz. Tiene 14 pines de I/O digitales (de los cuales 6 pueden usarse como salidas PWM), 8 entradas analógicas, un oscilador en placa, un botón de reinicio (reset), y agujeros para montar conectores de pines. Tiene conexiones para una batería de polímero de Litio e incluye un circuito de carga a través de USB. En el reverso de la placa tiene disponible un zócalo para módulos XBee.

El Arduino FIO está diseñado para aplicaciones inalámbricas. El usuario puede subir sus sketches con un cable FTDI o una placa adicional adaptadora Sparkfun. Además, si utiliza un adaptador de USB a XBee modificado (como el USB Explorador de XBee), el usuario puede subir sketches de forma inalámbrica. La tarjeta viene sin conectores pre-montados, permitiendo el uso de diversos tipos de conectores o la soldadura directa de los cables. La Figura \ref{table:Especificaciones Fio} tiene una tabla que resume sus características principales.

%-----------------------------------
%	SUBSECTION 2
%-----------------------------------
\subsection{Motores DC}
Este motor Pololu, Figura \ref{fig:compELO} f es muy pequeño (largo: 9.27 mm), por lo mismo usa muy poca corriente y tiene una caja de reducción metálica de 30:1. El eje tiene forma de D.

Pese a no ser la alternativa más económica, estos motores han sido muy difundidos para su uso en robótica por su bajo consumo y alto torque. La Figura \ref{fig:TablaElo} tiene una tabla que resume sus características principales.

%-----------------------------------
%	SUBSECTION 3
%-----------------------------------
\subsection{XBee}

Los dispositivos con los cuales interactuamos a diario poseen acceso a distintos tipos de sistemas de comunicación. Los \textit{Smartphone} cuentan con diversas antenas para conectarse a WIFI, Bluetooth, red celular,  por nombrar algunas. Estamos en un momento en que las comunicaciones son una de las claves tecnológicas que nos permiten cumplir con tareas que para nuestros abuelos eran imposibles de hacer en un día. Es claro que un sistema con varios o cientos de robots también necesitan comunicarse. Y en el caso particular de MODI, que por el momento no tiene sensores, es fundamental la información que se transmita para controlarlo. 

Para comunicar un robot con otro pueden usarse diversas técnicas. Dependiendo del área en particular que sea del interés del científico, los robot pueden comunicarse por medio de sensores de proximidad o tacto, para saber que hay otro cerca, pueden tener un sistema de visión artificial con una cámara a bordo que le permita “ver” el entorno al robot, con parlantes y micrófonos pueden emitir algún ruido que sea interpretado por su entorno, o lo mismo con sensores de otros tipos.

El robot que se construyó para hacer las pruebas cuenta con un sistema externo de visión artificial que permite obtener posición y orientación de cada robot. Esto es necesario ya que como se mencionó, no existe ningún sensor interno del robot que le pueda ayudar a ubicarse en el espacio. Esta información es transmitida a cada robot por medio del protocolo \textit{zigbee} implementado en los chip XBee de la empresa Digi, Figura \ref{fig:compELO} b. Esto permite comunicar un alto número de robots, con muy poco consumo energético y de manera relativamente simple. Se escogió usar visión artificial sobre el sistema para bajar los costos individuales de los robots y XBee por ser el estándar en la industria en redes de sensores inalámbricos. Cabe destacar que los robots que existen en el mercado no incluyen XBee, y los que lo traen lo hacen como accesorio. 

%-----------------------------------
%	SUBSECTION 3
%-----------------------------------
\subsection{Lipo Rider Pro}

El LiPo Rider Pro, Figura \ref{fig:compELO} a, es una versión mejorada del LiPo Rider, que es una fuente de poder para alimentar distintos gadget con energía verde. Esta placa permite alimentar con 5V distintos dispositivos. Puede obtener energía del sol o por medio de inducción magnética. También puede cargar un SmartPhone.

La Lipo Rider carga una batería de Litio Polímero. Posee una alta densidad de carga, 2000mAh a 3.7V. Figura \ref{fig:compELO} a.

%-----------------------------------
%	SUBSECTION 4
%-----------------------------------
\subsection{Motor Driver 1A Dual TB6612FNG}
El TB6612FNG motor driver puede controlar hasta dos motores DC, a una corriente constante de 1.2A (3.2A peak). Dos señales de entrada (IN1 y IN2) puede ser usadas para controlar el motor en una de cuatro modos de funcionamiento - CW, CCW, short-brake, y stop. Cada una de las salidas hacia los motores (A y B) se puede controlar separadamente, la velocidad de cada motor es controlada vía PWM con una frecuencia de hasta 100khz. El pin STBY debe colocarse en High para que el motor salga del estado Standby. Figura \ref{fig:compELO} g.

\begin{figure}[htbp]
	\centering
		\includegraphics[width=0.8\textwidth]{./Figures/MODI/comparacionElo.png}
		\rule{35em}{0.5pt}
	\caption[Tabla caracteristicas mas importantes de los componentes Eléctronicos]{Resumen de las características más importantes de los componentes electrónicos que tiene MODI.}
	\label{fig:TablaElo}
\end{figure}


%-----------------------------------
%	SECTION 3
%-----------------------------------
\section{Componentes mecánicos}

Como se comentó en la sección \ref{cap:impresora3D}, las piezas del robot fueron diseñadas para ser construidas por una maquina de prototipado rápido del tipo FDM y se tener en cuenta la tolerancia mecánica de las piezas. En especial hay dos piezas: la rueda y el chasis, que se unen al motor. Esta unión debe quedar con un ajuste mecánico Forzado Medio \footnote{http://campuscurico.utalca.cl/\~ fespinos/Ajustes\%20y\%20tolerancias\%20mecanicas.pdf}, para que estas uniones queden fijas y las ruedas esten en el eje correspondiente.

Está formado por cuatro partes:


\begin{itemize}
\item Chasis, debe cumplir con la tarea de contener toda la electrónica y motores
\item Ruedas, permiten el desplazamiento y poseen dos canales para poner o-ring y así aumentar el roce con la supercifie.
\item Accesorio, tiene un sistema de anclaje al chasis que permite tener distintas carcasa y sensores.
\item Logo,  sujeta parte de la electrónica contenida en el Chasis y un LED RGB.
\end{itemize}

La pieza que más demora en construirse es el Chasis, que son 108 minutos en una Replicator 1 de MakerBot.

\begin{figure}[htbp]
	\centering
		\includegraphics[width=\textwidth]{./Figures/MODI/piezas.jpg}
		\rule{35em}{0.5pt}
	\caption[Piezas 3D]{Las cinco piezas que componen el hardware mecánico de MODI, estas son: el chasis, las dos ruedas, el accesorio y logo.}
	\label{fig:Render Piezas 3D}
\end{figure}


%-----------------------------------
%	SECTION 4
%-----------------------------------

\section{Locomoción}
Existen distintas áreas donde se utilizan robots, dependiendo de la tarea a realizar es la movilidad que este debe tener. Algunos tienen complejos sistemas como la plataforma del proyecto Atlas (The Agile Anthropomorphic Robot) de Boston Dynamics, Figura \ref{fig:Atlas}. En nuestro caso, solo es necesario desplazarse sobre un superficie plana y por esto, en vez de tener costosos actuadores neumáticos, la solución más simple es contar con dos motores para poder tener un movimiento diferencial. Si se desea tener más información sobre los tipos de locomoción en robots con ruedas, ver esta pagina web\footnote{en.wikibooks.org/wiki/Robotics/Types\_of\_Robots/Wheeled}.


\begin{figure}[htbp]
	\centering
		\includegraphics[width=\textwidth]{./Figures/AtlasCADlr.jpg}
		\rule{35em}{0.5pt}
	\caption[Robot Atlas]{Atlas es un humanoide con gran movilidad, diseñado para poder usar herramientas humanas. Posee 28 grados de libertad con actuadores hidráulicos y la energía se le entrega por medio de un cable flexible. Imagen tomada de bostondynamics.com.}
	\label{fig:Atlas}
\end{figure}


Para moverse MODI tiene dos motores DC, ver figura \ref{fig:compELO} f. Estos tienen una caja de reducción 30:1 que permiten aumentar el torque y disminuir la velocidad. La Figura \ref{fig:DCMotor} muestra la característica de estos. 

\begin{figure}[htbp]
	\centering
		\includegraphics[width=0.8\textwidth]{./Figures/graficosMotores.png}
		\rule{35em}{0.5pt}
	\caption[Gráficos Motor DC]{a. Operación motores: Corriente, velocidad y torque. Imagen extraída de pololu.com, b. Control de velocidad PWM, se observa como disminuye la velocidad según el ancho del pulso pwm.}
	\label{fig:DCMotor}
\end{figure}

El control de velocidad de los motores se hace con PWM, que es una señal periódica a la cual se le modifica el ciclo de trabajo. El sentido de giro es controlado con el chip TB6612FNG \ref{fig:DCMotor} a. Los valores que genera el Arduino para control de los motores van desde -255 a 255, y para los distintos casos se puede ver la Figura \ref{fig:pwm} que indica los 3 movimientos que son: avanzar, avanzar hacia a la izquierda y girar en el eje. Aquí los numeros representan el periodo del ciclo de trabajo de la señal enviada por el Arduino al controlador de los motores. Se puede realizar un control mas fino de la posición usando la información de posicion del robot en un PID, se recomienda ver este video tutorial \footnote{http://www.youtube.com/watch?v=aE7RQNhwnPQ}. Para más información sobre PWM y este chip ver esta pagina web \footnote{http://www.pololu.com/docs/0J21/5.c}.

\begin{figure}[htbp]
	\centering
		\includegraphics[width=0.8\textwidth]{./Figures/MODI/pwm.png}
		\rule{35em}{0.5pt}
	\caption[Señal PWM]{Movimiento diferencial. A diferencia de un auto, MODI tiene dos motores que giran de forma independiente. Las distintas velocidades y sentido de giro determinan el angulo para girar. Si se necesita solamente rotar los motores deben girar en sentidos opuestos.}
	\label{fig:pwm}
\end{figure}

%-----------------------------------
%	SECTION 5
%-----------------------------------
\section{Implementación}
La construcción de MODI fue motivada por una aplicación concreta, que es tener un grupo de robots controlables de forma independiente para hacer estudios del comportamiento del grupo. Para esto fue necesario montar un sistema simple con una cámara cenital que vigila  la posición y orientación de cada uno.
%-----------------------------------
%	SECTION 6
%-----------------------------------

\subsection{Setup}

Para investigar el comportamiento colectivo de un grupo de robots, en el mundo real sin hacer uso de simuladores, es necesario contar con un lugar físico donde poder activar los robot. Además para simplificar cada uno de los robots, estos no tienen sensores internos por lo que hay una cámara montada sobre el plano de movimiento de estos, para hacer Seguimiento Visual. Esta información de coordenadas llega a un computador servidor que le indica a cada robot como moverse.
\begin{figure}[htbp]
	\centering
		\includegraphics[width=\textwidth]{./Figures/setup.png}
		\rule{35em}{0.5pt}
	\caption[Setup Enjambre MODI]{Setup a montarse para hacer estudios de grupos de robots.}
	\label{fig:setup}
\end{figure}
%-----------------------------------
%	SUBSECTION 6
%-----------------------------------

\subsection{Software}
Software, el cerebro del robot
Gracias a un puerto serial escuchando las instrucciones para mover los motores del robot, es bastante sencillo controlar a MODI desde el computador usando cualquier lenguaje que tenga una biblioteca para uso de puertos seriales.

A continuación algunos ejemplos de como controlar a MODI. El puerto serie en MODI tiene configurada una velocidad de 115200 y esta esperando los caracteres ‘w’,’a’,’d’ y ‘s’, como Adelante, Izquierda, Derecha y Atrás respectivamente. El código de MODI, el Firmware, se encuentra en el Git de MODI en Github.


\subsection{Tracking 2D}
Al igual que los seres vivos un robot necesita \textit{sentidos} o algún sistema sensorial para poder interactuar con su entorno. Estos pueden ser sensores IR para detectar objetos cercanos, cámaras o LASER para hacer un mapa del entorno o simplemente un pulsador que sea presionado cada vez que el robot colisiona. Para bajar costos, MODI actualmente no cuenta con sensores “onBoard”. Cada robot es comandado por un sistema que le dice donde donde está él, dónde están los demás y los límites del área de trabajo. Se diseñó de esta manera para simplificar la programación de cada robot.

reacTIVision \cite{kaltenbrunner2007reactivision} es un framework open source, cross-platform, para realizar tracking rápido y robusto de tags fiduciales pegadas a objetos físicos. Ha sido diseñado como un toolkit para el desarrollo rápido multi-touch interactive surfaces. Este Framework fue desarrollado por Martin Kaltenbrunner y Ross Bencina en el Music Technology Group de la universidad de Pompeu Fabra en Barcelona, España. reacTIVision fue diseñado como sensor principal de la Reactable, un sintetizador modular tangible que ha marcado los standards en lo que a aplicaciones multitouch se refiere. Esta alternativa debe ir acompañada de otros elementos, como la cámara y el computador en el cual se ejecuta el software. 


\begin{figure}[htbp]
	\centering
		\includegraphics[width=0.8\textwidth]{./Figures/MODI/fiducial.png}
		\rule{35em}{0.5pt}
	\caption[Fiduciales usados como tag en reacTIVision]{Códigos fiduciales utilizados en reacTIVision. Estos códigos aunque parecen algo extraños para los seres humanos están diseñados de tal forma que es muy fácil diferenciar unos de otros y además obtener su orientación y posición en un software de análisis de imágenes. Se tiene un total de 216 configuraciones posibles de códigos.}
	\label{fig:Fiducial}
\end{figure}



%-----------------------------------
%	SECTION 7
%-----------------------------------

\subsection{Bill Of Material}

Con un precio total de ~180 USD, figura \ref{fig:BOM}, MODI se propone como una alternativa económica para construir un robot. Pensando en comprar desde Chile, son tres las tiendas en que se cotizaron los componentes. Es posible comprar todo desde internet. Para más detalles sobre la Lista de materiales ver \footnote{http://kitbom.com/otrab/modi} 
\begin{figure}[htbp]
	\centering
		\includegraphics[width=\textwidth]{./Figures/MODI/kitbom.png}
		\rule{35em}{0.5pt}
	\caption[Bill Of Materials]{Lista de materiales a comprar para construir un robot MODI, la versión online de este BOM se puede ver en http://kitbom.com/otrab/modi}
	\label{fig:BOM}
\end{figure}
